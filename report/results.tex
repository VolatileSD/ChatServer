The first thing a user sees after connection to our system is the Authentication page.
    
The user has the option to log in, but only after creating an account. He can also remove an existing account. The system catches all this misuse alternatives and helps the user to the right path.

After the user is logged in, it is automatically in a chat room, the Main room, which is the default room of the service. In the simple text-based client, the user can type \emph{:help} to see the available commands. The GUI client has a panel that emulates just that: a text-box to write chat messages; a text area where the chat messages show up; a list of existing rooms or online users in a room; a button to go to his inbox; and, if it is an admin user, a button to go to admin options.

If the users clicks in the inbox, it can view the history of previous conversations with other users (:inbox command for the simple client lists only messages received) or send a private message to a user it never did before (:private user message for the simple client).

If an admin user clicks on the admin button option, it finds a panel where he can add or delete a room and give or take admin privileges to other users. 

In summary, all predictable error behaviour by the user is supported and information is displayed. During the implementation, other possible bugs were tested and, the ones found, were corrected.